\documentclass{article}

\usepackage[hidelinks]{hyperref}
\usepackage[sorting=none]
{biblatex}
\usepackage{geometry}
\addbibresource{reference.bib}
\geometry{a4paper, top=2cm, left=25mm, right=25mm, bottom=2cm}

\title{\textbf{Transformer Powered Two Level Matrix Factorization for Recommender Systems}} 

\author{Ricardo Plesz}
\date{August 21, 2021}

\begin{document}
\pagenumbering{arabic}
    
\maketitle

In the era of Big Data it becomes increasingly difficult to find information that is relevant to oneself. 
Recommender systems are used to cope with this flood of information: 
they do this by providing personalized subsets of information to the user. 
Today, they are used in a wide variety of domains with great success. For instance, $80\%$ of the
streaming time on Netflix is influenced by the recommender system Netflix operates \cite{gomez2015netflix}. 
\\

Online-argumentation is a domain where the user's opinion is influenced strongly by the
specific arguments he faces. Problems such as filter bubbles or arguments that
are not relevant to a specific user can be mitigated by using a recommender system
which provides suitable arguments to the user, depending on the objective of the recommender system.\\
In the original paper, data from over 600 individuals on 900 arguments at different points in time was collected \cite{HowIArgue}.
The goal was to provide a dataset that can be used to evaluate algorithms that predict how persuasive 
an argument is to a specific user.\\
Three tasks that use the known user-argument interaction data were presented to predict ratings of arguments 
by users that were collected at later points in time:
\begin{itemize}
    \item Predicting a user's conviction by an argument (binary classification)
    \item Predicting the strength of the conviction for an argument (multiclass classification in the range $[0,6]$)
    \item Predicting three convincing statements for a specific user
\end{itemize}
In this thesis I will focus on the first two tasks.
In order to obtain reference performances for such an algorithm, two baseline algorithms
were presented: a simple majority voter and a more sophisticated nearest-neighbor-algorithm.
The goal of this thesis is to implement an algorithm that exceeds the performance of these two 
baseline algorithms on the provided dataset.
\\

In the original paper the linguistic properties of the arguments were not exploited to make predictions \cite{HowIArgue}. 
The arguments were used to build an argumentation tree to eventually obtain numerical similarity
values for the argumentation behaviour of users \cite{brenneis2020much}. Hence, there were no content-based elements involved
in the predictions.\\ 
In the first step of this bachelor thesis I will re-implement a Two-level Matrix Factorization (TLMF) algorithm to improve upon the presented baseline algorithms \cite{li2016two}.
In the first level of the TLMF, semantic similarities of arguments are computed by applying a weighted textual matrix factorization (WTMF) that utilizes a term-document matrix.
In the second step this information is integrated into another matrix factorization to calculate the
user and item latent factors from which the original user-item matrix can be approximated.\\
In the second step of this bachelor thesis I will extend the TLMF algorithm by calculating the semantic similarities between arguments using 
a pre-trained BERT-model \cite{devlin2018bert}.\\
In the third step of this bachelor thesis I will implement and train a BERT-model from scratch. The BERT-model will be pre-trained on different datasets and tasks to build a language model 
capable of applying word-level-knowledge as well as sentence-level-knowledge. To further strengthen the latter, I will apply a novel symmetric sentence prediction pre-training 
task where also Previous-Sentence-Prediction (PSP) is applied instead of only applying the usual Next-Sentence-Prediction (NSP) \cite{xu2020symmetric}.
This pre-training task will tackle the problem of order-sensitivity of the BERT-architecture as well as forcing the model to learn deep contextual, semantic relations instead
of relying on high-level semantic concepts such as topics to distinguish between different documents.\\
In the last step of this bachelor thesis I will make use of the Augmented Sentence-BERT (AugSBERT) transformer architecture to incorporate
in-domain knowledge of the arguments' topic into the semantic similarity prediction of argument pairs \cite{thakur2020augmented}.
The AugSBERT-architecture applies both, a pre-trained, cross-encoder BERT-model as well as a siamese, bi-encoder Sentence-Bert (Sbert)-model \cite{reimers2019sentence}.
The pre-trained Bert model will soft-label the argument data, such that the siamese Sbert-architecture can be fine-tuned
on this synthetic data to boost performance on predicting the similarity of arguments of the known domain.
In this context, new sampling algorithms will be evaluated to create a balanced, augmented and soft-labeled dataset to fine-tune the
bi-encoder Sbert-model on. 

\printbibliography

\end{document}