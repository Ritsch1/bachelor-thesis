\documentclass{article}

\usepackage[hidelinks]{hyperref}
\usepackage[sorting=none]
{biblatex}
\usepackage{geometry}
\addbibresource{reference.bib}
\geometry{a4paper, top=2cm, left=25mm, right=25mm, bottom=2cm}

\begin{document}

\pagenumbering{arabic}

%Front Page
\title{\vspace{-0.0cm} Abstract Proposal\\
Bachelor Thesis
\\}
\author{Ricardo Plesz}
\date{July 29, 2021}
\maketitle

In the era of Big Data it becomes increasingly difficult to find information that is relevant to oneself. 
Recommender systems are used to cope with this flood of information: 
they do this by providing personalized subsets of information to the user. 
Today, they are used in a wide variety of domains with great success. For instance, $80\%$ of the
streaming time on Netflix are influenced by the recommender system Netflix operates. \cite{gomez2015netflix}
\\

Online-argumentation is a domain where the users' opinion is influenced strongly by the
specific arguments he faces. Problems such as filter bubbles or arguments that
are not relevant to a specific user can be mitigated by using a recommender system
which provides suitable arguments to the user-depending on the objective of the system.
In the original paper \cite{HowIArgue} data from over 600 individuals on 900 arguments at different points in time was collected.
The goal was to provide a dataset that can be used to evaluate algorithms that predict how persuasive 
an argument is to a specific user.\\
Three tasks were presented that use the known user-argument interaction data to predict ratings of arguments 
by users at later time points:
\begin{itemize}
    \item Predicting a user's conviction by an argument (binary classification)
    \item Predicting the strength of the conviction (multiclass classification in the range $[0,6]$) for an argument
    \item Prediction three convincing statements for a specific user
\end{itemize}
In this thesis, I will focus on the first two tasks.
In order to obtain reference performances for such an algorithm, two baseline algorithms
were presented: A simple majority voter and a more sophisticated nearest-neighbor-algorithm.
The goal of this thesis is to implement algorithms that exceed the performance of these two 
baseline algorithms on the provided dataset.
\\

In the original paper \cite{HowIArgue} the linguistic properties of the arguments were not considered to make predictions. 
The arguments were used to build an argumentation tree to eventually obtain numerical similarity
values for the argumentation behaviour of users \cite{brenneis2020much}. So there were no content-based elements involved
in the predictions. In this bachelor thesis I will implement a Two-level Matrix Factorization (TLMF) algorithm that was published in $2016$ \cite{li2016two}. 
In the first step of the TLMF, semantic similarities of arguments are computed using a weighted textual matrix factorization (WTMF). 
In the following step this information is integrated into another matrix factorization to calculate the
user and item latent factors from which the original user-item matrix can be approximated.
Different techniques from the field of Natural Language Processing (NLP) e.g. Transformers \cite{vaswani2017attention}
will be applied to compute similarities between arguments and to modify the WTMF-algorithm with the aim to optimize the prediction performance on the 
provided dataset.

\printbibliography

\end{document}